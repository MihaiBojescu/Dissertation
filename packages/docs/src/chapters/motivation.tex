\chapter*{Motivație} 
\addcontentsline{toc}{chapter}{Motivație}

Generarea de muzică este un domeniu mai puțin explorat în procesarea limbajului natural, comparativ cu generarea de text, deși acesta are implicații critice în domeniile artă, educație și tehnologie. În procesarea de text, deseori pot fi realizate și sarcini adiționale, precum transferarea de stil sau personalizarea rezultatelor după instrucțiunile utilizatorului. Realizarea acestor sarcini este facilitată de modelele de tip ``\textit{Transformer}'', un model ce a reușit să revoluționeze domeniul procesării limbajului natural și să avanseze modelele generative de imagini. Deseori, procesarea textului și a secvențelor audio folosesc procedee diferite de preprocesare, însă modele asemănătoare. În această lucrare propun un model ce folosește tehnici folosite în procesare de text, imagini și secvențe audio, model ce încearcă să producă secvențe audio muzicale credibile. 